\subsection{Panel Discussion on ``Software Engineering \& Deployment''}

\subsubsection{Panelists}

\begin{itemize}
\item Andrew Bauer (Kitware -- Clifton Park, US, andy.bauer@kitware.com)
\item Jens Krüger  (Universität Duisburg-Essen, jens.krueger@uni-due.de)
\item Matthew Larsen (Lawrence Livermore National Laboratory, larsen30@llnl.gov)
\item Kenneth Moreland (Sandia National Laboratories -- Albuquerque, kmorel@sandia.gov)
\end{itemize}

%\subsubsection{Statements}
%
%\paragraph{Joan R. Access}
%
%This is the statement of Joan R. Access.
%
%\paragraph{Jane Doe}
%
%And this is the statement of Jane Doe.
%
%\paragraph{John Q. Open}
%
%John Q. Open -- believe it or not -- also gave a statement.

\subsubsection{Summary of Discussion}

\wgpar{Background \& Motivation.} Software is the means by which theoretical and methodological techniques are made practical and useful to a user community.  Depending on a researcher’s priorities, their target user-community can vary dramatically.  For example, an academic researcher can have a target user-community of a) application scientists/engineers or b) their students.  Each of these user communities places different requirements on the quality, maturity, and support model of the resulting software product.  In the first scenario, robust, user-friendly, production-quality software is often required for adoption.  In the second scenario, the maturity required of the software product is typically significantly lower (e.g., the software is a learning tool, that the students are meant to build off of in their studies).  

Software can be made accessible to users via a number of mechanisms: commercial, open-source, and via direct hand-off (but with no formal licensing/use agreements). Once software is deployed, user-support model requirements will vary by user-type. 

Software also serves as a means for a researcher to establish an identity (analogous to publications).

\wgpar{Challenges.} Many of the software challenges highlighted by the workshop attendees are not necessarily unique to the in situ research-to-development pipeline, but are exacerbated by the interdisciplinary nature of the work.  Challenges are often due to discrepancies in priorities (and resulting expectations) between an in situ tools researcher and their target user-community. These discrepancies can be minor (e.g., missing a feature that might be easily be added), to significant (researcher has a prototype implementation with a minimal user-support model, whereas the user-community may want a production-ready tool). However, challenges can also be due to language barriers between tool and user communities (e.g., unclarified assumptions, use of terms that mean different things to the two communities, jargon).   

The following examples illustrate common discrepancies in priorities between in situ tools developers and their target user communities:

\begin{itemize}
\item \textbf{\sffamily Requirements.} Scientists and engineers often prefer specialized tools, tailored to their application, over general tools.  This is due, in large part, to their ease of use. In an in situ workflow,  these preferences become strict requirements due to the increase in overall code-complexity and the implications of a tool’s downstream dependencies. Technical user community concerns include stability of application programming interfaces (API), code and/or compile-time bloat, and code reliability. In situ tool developers, on the other hand, prefer to create general purpose tools, which promote sustainable code management and simplify (from the point of view of the tool developer) the process of supporting multiple target user-communities.

\item \textbf{\sffamily Advertising and Adoption.} Communication barriers can arise between user and tool communities in several ways that inhibit advertising. For example, different communities use of similar terms to mean different things.  Jargon and acronyms can also serve as a barrier to adoption.   Communication issues can be further exacerbated when competing tools exist that provide overlapping  functionality. Not only can this cause friction and competition within a particular tool community, it can make it difficult for user communities to discern which of the competing tools to use.  Aspects that may further inhibit adoption include, e.g., concerns over which tool will have greater longevity, lack of clarity regarding tradeoffs between the tools, and uncertainty about downstream dependencies.  

\item \textbf{\sffamily Accessibility.} Researchers have different mechanisms to make their tools available to their target user-community.  Commercial tools come at a financial cost to users, which provides a higher barrier-to-entry.  For tool developers a primary benefit of commercialization is the financial means to fund code development and user-support. In contrast, releasing an open source tool imposes a lower barrier-to-entry for users.  However, tool developers relinquish some control of ownership/identity. Furthermore, open source release of a code does not provide a direct funding mechanism for further development and user-support. We note that some scientists and engineering user communities have accessibility requirements on tools used within their workflows (e.g., must be open source).
\end{itemize}

\wgpar{Potential solutions.} In spite of the numerous challenges, workshop attendees identified several suggestions for the path forward.  Several of these focus on addressing concerns raised by the user community regarding in situ tool APIs, their stability, and their intrusion into user application code.   The first suggestion is longer-term, nebulous, and strategic: think radically ``outside the box'' to facilitate ``application unawareness.''  An analogy was made to cell phone chargers.  Every time a cell phone company changes the form factor of its charging plug, it is disruptive to their user community.  Recently, innovative wireless phone charging equipment has provided a revolutionary solution to this problem. Is there a similar revolutionary technology that can address the ``tower of Babel'' API complexity issue? 

Other suggestions to address API concerns are nearer-term, specific, and more tactical in nature:

\begin{itemize}
\item Identify the lowest common denominator that is general, but still allows for specialization for specific applications (e.g., something analogous to the POSIX file system for post-hoc analysis use case).  
\item Engage external tools communities, for example, those developing containers (e.g., Docker, Kubernetes)  and package managers (e.g., SPACK, LMOD). 
\end{itemize}

\noindent
Many of the advertising and adoption concerns raised were due to communication concerns.  Potential solutions for these issues were raised but are discussed in greater detail in the Teaming and Pipeline subsection.

