\subsection{Panel Discussion on ``Programmatic \& Funding Issues / Interdisciplinary / Pipeline''}

\subsubsection{Panelists}

\begin{itemize}
\item Peer-Timo Bremer (Lawrence Livermore NationalLaboratory, bremer5@llnl.gov)
\item Charles D. Hansen (University of Utah, hansen@cs.utah.edu)
\item Ingrid Hotz (Linköping University, ingrid.hotz@liu.se)
\item Bruno Raffin (INRIA -- Grenoble, bruno.raffin@inria.fr)
\item Alejandro Ribes Cortes (EDF -- Paris, alejandro.ribes@edf.fr)
\item Han-Wei Shen (Ohio State University -- Columbus, shen.94@osu.edu)
\end{itemize}

\subsubsection{Summary Discussion on Programmatic \& Funding Issues}
\label{sec:funding}
\begin{refsection}

\wgpar{Background and Motivation.} In a research-to-development pipeline, funding for the various stages of software development is often provided by multiple different funding agencies, each with very different ``success criteria.''  For example, many of the funding sources that promote in situ tools research are focused on theory and may relegate software development activities to prototypes only. However, user communities often require mature and reliable software, and their associated funding sources typically do not support software development and user support.  Commercialization and collaborations with large corporations are other means of funding an in situ research-to-production pipeline.

\wgpar{Challenges.} One of the predominant challenges identified by attendees is the dearth of sustained funding sources for software development, testing, and deployment activities aimed at bringing fundamental research tools into a production-quality state.  As noted above, there are distinct funding agencies that support both the in situ tools and user communities.  However, each of these funding agencies’ success criteria are narrowly focused within their respective communities.  Those funding sources aimed at bridging communities are few, and often transient in nature (limited duration).  Those that provide more sustained funding are focused on very narrow user communities (e.g., DOE/NNSA/ASC funding).  

Workshop attendees also noted some disconnects between funding program manager priorities and  the priorities of in situ tools researchers.  Some funding agencies have moved away from a focused emphasis on in situ processing, and are now looking to invest in newer research trends, such as machine learning and big data.  This is in large part due to the hype and public appeal surrounding these trends in industry.  However, this also highlights a potential communication disconnect between the in situ tools community and funding agency program managers on the importance of continued investment in an in situ research-to-production pipeline.

While commercialization and collaboration with industry are another source of funding, many workshop participants noted that non-disclosure agreement (NDA) requirements are often in direct conflict with fundamental tools researcher goals of publishing their research. 

Lastly, some in situ tools researchers noted difficulty in gaining access to high performance computing resources on HPC systems for user support build and test development operations efforts.

\wgpar{Potential Solutions.} We begin this section by highlighting some early success stories in addressing  funding issues:

\begin{itemize}
\item \textbf{\sffamily HPC resource allocation.}  In the United States,  HPC resource allocation may be easier than it appears at some HPC facilities. Many in situ tools researchers do not pursue resource allocation because acquisition processes appear to be targeted solely at application developers.  Although not well-advertised, there are compute cycles available to the tools communities and tools researchers have been successful in acquiring compute cycles by directly contacting facilities support staff to better understand request protocol options.
\item  \textbf{\sffamily Institution-specific success stories.}
\begin{itemize}
\item KAUST University has a support model for software development and deployment.   Research faculty can work with software developers who are paid by the University to mature research tools into production capabilities.  
\item EDF was highlighted as a company who funds work that spans the in situ research-to-production pipeline.  
\end{itemize}
\item \textbf{\sffamily Recent changes to funding agency reporting criteria.} Some academic and fundamental research funding agencies have made changes to their reporting/success criteria. Specifically, the agencies are beginning to ask about data artifacts and software sustainability plans.  While additional funding is not being provided to support these activities, this is an important first step in that direction.
\end{itemize}

\noindent
In addition to early successes, suggestions for path forward highlight the need for increased alignment between in situ pipeline researchers and potential funding sources:

\begin{itemize}
\item \textbf{\sffamily Academic and/or government funding agencies.} Academic researchers do not always have a clear understanding of a funding agency’s program manager roles and responsibilities (which may differ significantly across agencies).  Researchers could benefit from an understanding of funding agency terminology (e.g., technology readiness level (TRL)~\cite{TRL} to describe software maturity in a standardized way).  A deeper, understanding of a funding agency’s own success criteria and requirements would enable researchers to help shape a program manager’s research focus, as well as to help program managers  advocate for additional  funding for their programs.
 \item \textbf{\sffamily Industry collaborations.}  As we establish a research-to-production pipeline, it is clear that the community has not yet identified which universal, baseline services industry partners should develop and maintain.  For example, BLAS was highlighted as a former fundamental research activity that HPC industry providers now maintain across HPC system procurements.  What is the analogous BLAS for the in situ tools community? How should the in situ tools community identify this division of labor with industry partners?
\end{itemize}

\printbibliography	
\end{refsection}


\subsubsection{Summary Discussion on Teaming and Staffing Pipeline}

\wgpar{Background and Motivation.} 
The establishment of an in situ research-to-production pipeline is interdisciplinary, involving three different research communities (application scientists and engineers, high-performance computing experts, and visualization scientists)  across academia, industry, and research laboratories.  Universities are the staffing pipeline source, providing the next generation of researchers who will be joining these three communities. 

\wgpar{Challenges.} 
This subsection summarizes attraction, retention, and teaming challenges that negatively impact the establishment of an in situ research-to-production pipeline.

\begin{itemize}
    \item \textbf{\sffamily Staffing Pipeline.} While science-based research activities draw some students, the competition from industry is strong.  This is due to salary potential in industry, and the hype of data science and machine learning, which compete directly with the in situ tools community.  Consequently, students at universities are showing more interest in curriculum that prepares them for industry careers in data science, diminishing the in situ staffing pipeline. These challenges are exacerbated by losses in funding for universities, making it difficult, if not impossible, for professors and their students to collaborate on interdisciplinary research projects with laboratories and industry partners.  The influx of data science and machine learning career options (and their associated salaries), pose a retention challenge as well, which particularly impacts  research laboratory staffing.
    \item \textbf{\sffamily Teaming.} Many teaming challenges have been alluded to already throughout this report because communication and trust are central to effective interdisciplinary work. Specifically, when a team does not effectively communicate, this exacerbates any technical and logistical issues.  From a communication perspective, a number of issues impede effective collaboration.  For example, different research communities may use similar terminology, but mean different things.  Furthermore, the use of technical jargon and/or acronyms can cause additional confusion. Different communication styles and different priorities,  can lead to misunderstandings that undermine trust, further hampering collaboration.
\end{itemize}

\wgpar{Potential Solutions.}
  Workshop attendees identified suggestions, which mostly comprise teams learning and employing best practices in communication and psychology.
  
\begin{itemize}
    \item \textbf{\sffamily Staffing Pipeline.} suggestions on staffing pipeline center around understanding what motivates people to draw and maintain a healthy pool of interdisciplinary team members.  These include:
    \begin{itemize}
        \item Ensure internship mentors are educated in best practices  
        \begin{itemize}
            \item Understand that the experience of the internship is equally as important as the output from an attraction/retention perspective.
            \item  Meet regularly with both mentees and their professors.
            \item  Provide students with leadership opportunities.
            \item  Labs, industry, and funding agencies need to financially support recurring internships to establish long-standing relationships with potential employees.
        \end{itemize}
        \item Communicate with funding agencies the importance of sustained funding for in situ for University partners to maintain a healthy staffing pipeline (see ~\ref{sec:funding}).
        \item Work with human resource staffing partners at Universities and Laboratories to make salaries as competitive as possible with industry.
        \item Formalize recruiting ``sales pitch'': why should students want to pursue a career pursuing in situ pipeline work? Some of this may be technical, but other recruiting factors could be a draw: e.g., impact, team, work-life balance, etc.
        \item Figure out how to capitalize on data science and machine learning to draw students towards science and/or mission-impact application domains, which overlaps with thrust from Section~\ref{sec:BigData}.
    \end{itemize}
    \item \textbf{\sffamily Teaming.} Interdisciplinary are likely to be diverse along several axes, including technical field, institutional, and social/communication style.  Suggestions for teaming centered around embracing diversity and inclusion best practices to build and sustain effective teams:
\begin{itemize}
    \item Communicate clearly and without jargon.
\item  Avoid assumptions: understand your own as well as your team members’. 
\item Understand your team member’s priorities and success criteria.
\item Find win-win situations: identify and agree upon mutually agreeable successful outcomes.  Do this early in the teaming process and revisit on regular intervals.
\item Education and tools: 
\begin{itemize}
    \item Provide team leads education on coaching.  
\item Educate all team members on social styles (e.g., DISCS, Myers-Briggs) and communication (e.g., convergent vs divergent thinking).  
\item Embrace associated tools chains.
\end{itemize}
\item Choose teams wisely: avoid working with people who do not abide by communication and teaming best practices.
\item Communicate about norms.  This includes what data is private and what is public.  Mailing lists, social media and some cloud services may be unwelcoming environments for some potential team members.
\item Understand and manage the costs (time, financial investment, education) of collaboration.

\end{itemize}
\end{itemize}