In addition to identifying ten research topics, workshop attendees agreed upon the importance of establishing a pipeline to bring fundamental, theoretical in situ research into a production environment (i.e., sustainable, user-friendly, robust, high-quality software tools that are actively used by science and engineering application teams).  
%
Attendees also noted, however, that many pervasive, cross-cutting issues impede community progress in establishing an in situ research-to-production pipeline.  
%
These issues are due, in large-part, to the interdisciplinary nature of the pipeline. 
%
Specifically, each individual researcher within an interdisciplinary team has a different set of priorities or “definitions of success” that shapes the focus of their work.  
%
Individual priorities are defined by many factors, including research community, personal interests, funding sources, and/or performance measures at their institution (academia, industry, research laboratory).  
%
In order to build a successful in situ research-to-production pipeline,  interdisciplinary team members need to resolve or ``bridge'' these very different priorities, some of which are outside of the team’s direct control. Workshop attendees identified a number of bridging issues and classified them into three categories: 1) Software Engineering and Deployment, 2) Funding, and 3) Teaming and Pipeline.   
%
The first issue was discussed in the first panel, and the latter two were discussed in the second panel.
