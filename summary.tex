%This summary summarizes the outcomes of  our seminar. The seminar focused on\begin{itemize}
%\item important issues,
%\item relevant problems, and
%\item adequate solutions.
%\end{itemize}

%As a major result from the seminar, the following problems have been identified: 
%\begin{enumerate}
%\item The problem of writing a brief, but concise executive summary.
%\item The problem of collecting all abstracts from talks.
%\item The problem of preparing summaries from working groups, open problem sessions, and panel discussions.
%\end{enumerate}

\noindent
The workshop identified ten challenges for in situ processing that require significant research.
%
These challenges were identified by spending the first day of the workshop with participants giving short
presentations on their experiences with in situ processing, with a special focus on unsolved problems.
%
The participant perspectives were then organized into the ten research challenges.
%
Over the following days, 
 sub-groups  discussed each of the ten challenges and then presented the key points of their discussions to the group and received feedback.
%
Shortly after the workshop, the leaders of each sub-group wrote summaries for its associated research challenge; these summaries are the basis of this report.

\medskip\noindent
The ten challenges identified by our participants were:

\begin{itemize}
    \item Data quality and reduction, i.e., reducing data in situ and then exploring it post hoc, which is likely the form that will enable exploration of large data sets on future supercomputers.
    \item Workflow specification, i.e., how to specify the composition of different tools and applications to facilitate the in situ discovery process.
    \item Workflow execution, i.e., how to efficiently execute specified workflows, including workflows that are very complex.
    \item Exascale systems, which will have billion-way concurrency and disks that are slow relative to their ability to generate data.
    \item Algorithmic challenges, i.e., algorithms will need to integrate into in situ ecosystems and still perform efficiently.
    \item Use cases beyond exploratory analysis, i.e., ensembles for uncertainty quantification and decision optimization, computational steering, incorporation of other data sources, etc.
    \item Exascale data, i.e., the data produced by simulations on exascale machines will, in many cases, be fundamentally different than that of previous machines.
    \item Cost models, which can be used to predict performance before executing an algorithm and thus be used to optimize performance overall.
    \item The convergence of HPC and Big Data for visualization and analysis, i.e., how can developments in one field, such as machine learning for Big Data, be used to accelerate techniques in the other?
    \item Software complexity, heterogeneity, and user-facing issues, i.e., the challenges that prevent user adoption of in situ techniques because in situ software is complex, computational resources are complex, etc.
\end{itemize}

\noindent
From group discussion, two other important topics emerged that do not directly lead to open research questions, but rather are concerned with effective organization of the often highly interdisciplinary research into in situ techniques. To address these, two panels were held to facilitate effective discussion. Finally, the workshop featured technical presentations by participants on recent results related to in situ visualization.