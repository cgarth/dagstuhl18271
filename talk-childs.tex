
\abstracttitle{The In Situ Terminology Project}
\abstractauthor[Hank Childs]{Hank Childs (University of Oregon, hank@uoregon.edu)}
\license

%\jointwork{}
%\abstractref[http://dx.doi.org/1234.12/Doe.DOI.34]{Doe, John; Doe, Jane, Sample talk abstracts used for Dagstuhl Reports, Journal of Seminar Documentation, 1:8, pp.~34--78.}
%\abstractrefurl{http://dx.doi.org/1234.12/Doe.DOI.34}

The term ``in situ processing'' has evolved over the last decade to mean both a specific strategy for processing data and an umbrella term for a processing paradigm. The resulting confusion makes it difficult for visualization and analysis scientists to communicate with each other and with their stakeholders. To address this problem, a group of approximately fifty experts convened with the goal of standardizing terminology. This presentation summarizes their findings and proposes a new terminology for describing in situ systems. An important finding from this group was that in situ systems can be described via multiple, distinct axes: integration type, proximity, access, division of execution, operation controls, and output type. This paper discusses these axes, evaluates existing systems within
the axes, and explores how currently used terms relate to the axes.

%\begin{thebibliography}{0}
%\bibitem{dagrep-manual} Schloss Dagstuhl -- Editorial Office,\textsl{The dagrep class}. Schloss Dagstuhl, Germany, 2011.
%\bibitem{dagrep-sample} John Q. Open; Joan R. Access, \textsl{Seminar Sample}, Dagstuhl Reports, 1:1, 1--8, 2011.
%\end{thebibliography}