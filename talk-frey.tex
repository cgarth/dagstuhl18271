\abstracttitle{Reduced Representation Tradeoffs, Dynamic Prediction and Adjustment}
\abstractauthor[Steffen Frey]{Steffen Frey (University of Stuttgart, steffen.frey@visus.uni-stuttgart.de)}

\license

%\jointwork{}
%\abstractref[http://dx.doi.org/1234.12/Doe.DOI.34]{Doe, John; Doe, Jane, Sample talk abstracts used for Dagstuhl Reports, Journal of Seminar Documentation, 1:8, pp.~34--78.}
%\abstractrefurl{http://dx.doi.org/1234.12/Doe.DOI.34}

Reduced representations for in situ visualization often exhibit tradeoffs between different quantities of interest, including data size, generation time, quality/accuracy, etc. These tradeoffs are steered via parameters, and can be adjusted dynamically during the execution to account for the variability in the data and the simulation process. The basis for informed adjustments are prediction models that estimate the impact of parameters on quantities of interest. In the talk, first Volumetric Depth Images (a reduced representation for volume data) and the involved tradeoffs are presented. Second, different approaches for online prediction and balancing from related visualization scenarios are outlined, and finally open research questions to bring these two parts together for in situ visualized are discussed.

%\begin{thebibliography}{0}
%\bibitem{dagrep-manual} Schloss Dagstuhl -- Editorial Office,\textsl{The dagrep class}. Schloss Dagstuhl, Germany, 2011.
%\bibitem{dagrep-sample} John Q. Open; Joan R. Access, \textsl{Seminar Sample}, Dagstuhl Reports, 1:1, 1--8, 2011.
%\end{thebibliography}