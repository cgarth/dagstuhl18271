\abstracttitle{Performance Modeling of In Situ Rendering}
\abstractauthor[Matthew Larsen]{Matthew Larsen (Lawrence Livermore National Laboratory, larsen30@llnl.gov)}
\license

%\jointwork{}
%\abstractref[http://dx.doi.org/1234.12/Doe.DOI.34]{Doe, John; Doe, Jane, Sample talk abstracts used for Dagstuhl Reports, Journal of Seminar Documentation, 1:8, pp.~34--78.}
%\abstractrefurl{http://dx.doi.org/1234.12/Doe.DOI.34}

With the push to exascale, in situ visualization and analysis will continue to play an important role in high performance computing. Tightly coupling in situ visualization with simulations constrains resources for both, and these constraints force a complex balance of trade-offs. A performance model that provides an a priori answer for the cost of using an in situ approach for a given task would assist in managing the trade-offs between simulation and visualization resources. In this work, we present new statistical performance models, based on algorithmic complexity, that accurately predict the run-time cost of a set of representative rendering algorithms, an essential in situ visualization task. To train and validate the models, we conduct a performance study of an MPI+X rendering infrastructure used in situ with three HPC simulation applications. We then explore feasibility issues using the model for selected in situ rendering questions.

%\begin{thebibliography}{0}
%\bibitem{dagrep-manual} Schloss Dagstuhl -- Editorial Office,\textsl{The dagrep class}. Schloss Dagstuhl, Germany, 2011.
%\bibitem{dagrep-sample} John Q. Open; Joan R. Access, \textsl{Seminar Sample}, Dagstuhl Reports, 1:1, 1--8, 2011.
%\end{thebibliography}