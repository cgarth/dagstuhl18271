\abstracttitle{Using VTK-m}
\abstractauthor[Kenneth Moreland]{Kenneth Moreland (Sandia National Labs -- Albuquerque, kmorel@sandia.gov)}
\license

%\jointwork{}
%\abstractref[http://dx.doi.org/1234.12/Doe.DOI.34]{Doe, John; Doe, Jane, Sample talk abstracts used for Dagstuhl Reports, Journal of Seminar Documentation, 1:8, pp.~34--78.}
%\abstractrefurl{http://dx.doi.org/1234.12/Doe.DOI.34}

One of the most critical challenges for high-performance computing (HPC) scientific visualization is execution on massively threaded processors. Of the many fundamental changes we are seeing in HPC systems, one of the most profound is a reliance on new processor types optimized for execution bandwidth over latency hiding. Our current production scientific visualization software is not designed for these new types of architectures. To address this issue, the VTK-m framework serves as a container for algorithms, provides flexible data representation, and simplifies the design of visualization algorithms on new and future computer architecture.

%\begin{thebibliography}{0}
%\bibitem{dagrep-manual} Schloss Dagstuhl -- Editorial Office,\textsl{The dagrep class}. Schloss Dagstuhl, Germany, 2011.
%\bibitem{dagrep-sample} John Q. Open; Joan R. Access, \textsl{Seminar Sample}, Dagstuhl Reports, 1:1, 1--8, 2011.
%\end{thebibliography}