\abstracttitle{Fast Fourier Transform in Solving Partial Differential Equations}
\abstractauthor[Benson Muite]{Benson Muite (University of Tartu, benson.muite@ut.ee)}
\license

%\jointwork{}
%\abstractref[http://dx.doi.org/1234.12/Doe.DOI.34]{Doe, John; Doe, Jane, Sample talk abstracts used for Dagstuhl Reports, Journal of Seminar Documentation, 1:8, pp.~34--78.}
%\abstractrefurl{http://dx.doi.org/1234.12/Doe.DOI.34}

The Fast Fourier Transform (FFT) is an algorithm used in the solution of many partial differential equations, primarily as a linear system solver. When the objective is the investigation of the properties of the partial differential equations, visualization is extremely important. In situ visualization helps avoid the io bottleneck. In situ frameworks make it easy to use supercomputers for scientific investigation of the properties of partial differential equations. 

In cases where the FFT is not scalable, other methods for solving linear systems of equations can be used. Most of these methods require a sparse matrix vector multiply. Benchmarks such as HPCG and Graph 500 are highly correlated because the primary kernel ``is a sparse matrix vector multiply''. A question of interest is what are the primary kernels/patterns of interest for in situ visualization which will become more important on next generation supercomputers.

%\begin{thebibliography}{0}
%\bibitem{dagrep-manual} Schloss Dagstuhl -- Editorial Office,\textsl{The dagrep class}. Schloss Dagstuhl, Germany, 2011.
%\bibitem{dagrep-sample} John Q. Open; Joan R. Access, \textsl{Seminar Sample}, Dagstuhl Reports, 1:1, 1--8, 2011.
%\end{thebibliography}

