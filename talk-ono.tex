\abstracttitle{Design of In Situ Framework for Time-Varying Data}
\abstractauthor[Kenji Ono]{Kenji Ono (Kyushu University, keno@cc.kyushu-u.ac.jp)}
\license

%\jointwork{}
%\abstractref[http://dx.doi.org/1234.12/Doe.DOI.34]{Doe, John; Doe, Jane, Sample talk abstracts used for Dagstuhl Reports, Journal of Seminar Documentation, 1:8, pp.~34--78.}
%\abstractrefurl{http://dx.doi.org/1234.12/Doe.DOI.34}

This talk gives an in situ framework designed to provide flexible data processing and visualization for time-varying data on an HPC system. The data staging approach used in the proposed framework utilizes the capabilities of the OpAS (Open Address Space) library, which enables asynchronous access, from outside processes, to any exposed memory region in the simulation side. A concept of temporal buffer is used to store the time-varying simulation results to execute interactive data processing and visualization. These features are expected to facilitate the decoupling of the simulation running time from the data processing and visualization execution time, and consequently to improve the flexibility for the in situ processing and visualization. As examples of this framework, we discuss a usage, ability, and code integration briefly using a CFD application for wind prediction on terrain.

%\begin{thebibliography}{0}
%\bibitem{dagrep-manual} Schloss Dagstuhl -- Editorial Office,\textsl{The dagrep class}. Schloss Dagstuhl, Germany, 2011.
%\bibitem{dagrep-sample} John Q. Open; Joan R. Access, \textsl{Seminar Sample}, Dagstuhl Reports, 1:1, 1--8, 2011.
%\end{thebibliography}