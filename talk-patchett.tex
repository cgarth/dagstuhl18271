\abstracttitle{Optimizing Scientist Time\newline Through In Situ Visualization and Analysis}
\abstractauthor[John Patchett]{John Patchett (Los Alamos National Laboratory, patchett@lanl.gov)}
\license

%\jointwork{}
%\abstractref[http://dx.doi.org/1234.12/Doe.DOI.34]{Doe, John; Doe, Jane, Sample talk abstracts used for Dagstuhl Reports, Journal of Seminar Documentation, 1:8, pp.~34--78.}
%\abstractrefurl{http://dx.doi.org/1234.12/Doe.DOI.34}

Simulation scientist time should be considered when designing simulation runs. The decision to produce artifacts, in particular what kind of artifacts, representing the simulation has implications on the time a scientist must spend to understand the simulation state. This talk relates observations of four techniques used by a simulation scientist during a set of simulations runs to study asteroid generated tsunami: lossless compression, resampling to image data, feature extraction using simple threshold operation, and maintaining a small amount of provenance as metadata. All of these have the potential of saving the scientist time while doing analysis at the expense of a small amount of computational time.

%\begin{thebibliography}{0}
%\bibitem{dagrep-manual} Schloss Dagstuhl -- Editorial Office,\textsl{The dagrep class}. Schloss Dagstuhl, Germany, 2011.
%\bibitem{dagrep-sample} John Q. Open; Joan R. Access, \textsl{Seminar Sample}, Dagstuhl Reports, 1:1, 1--8, 2011.
%\end{thebibliography}