\abstracttitle{Challenges for the visualization and analysis\newline of high-resolution simulation data }
\abstractauthor[Niklas Röber]{Niklas Röber (DKRZ Hamburg, roeber@dkrz.de)}
\license

%\jointwork{}
%\abstractref[http://dx.doi.org/1234.12/Doe.DOI.34]{Doe, John; Doe, Jane, Sample talk abstracts used for Dagstuhl Reports, Journal of Seminar Documentation, 1:8, pp.~34--78.}
%\abstractrefurl{http://dx.doi.org/1234.12/Doe.DOI.34}

With growing data sizes and simulation complexity, the processes of visualization and analysis become technically more difficult, but at the same time also more important. This particularly holds for weather and climate simulations that already produce tera- and petabytes of data for high-resolution simulation runs. This presentation shows examples of large data visualizations, devises workflows to handle massive amounts of data, as well as discusses the benefits and drawbacks for an in-situ visualization of the simulation data.

%\begin{thebibliography}{0}
%\bibitem{dagrep-manual} Schloss Dagstuhl -- Editorial Office,\textsl{The dagrep class}. Schloss Dagstuhl, Germany, 2011.
%\bibitem{dagrep-sample} John Q. Open; Joan R. Access, \textsl{Seminar Sample}, Dagstuhl Reports, 1:1, 1--8, 2011.
%\end{thebibliography}