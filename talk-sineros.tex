\abstracttitle{Toward Tuned to Terrific: \newline Parallel Particle Advection, I/O Optimization, and Deep Learning}
\abstractauthor[Robert Sisneros]{Robert Sisneros (University of Illinois at Urbana Champaign, sisneros@illinois.edu)}
\license

%\jointwork{}
%\abstractref[http://dx.doi.org/1234.12/Doe.DOI.34]{Doe, John; Doe, Jane, Sample talk abstracts used for Dagstuhl Reports, Journal of Seminar Documentation, 1:8, pp.~34--78.}
%\abstractrefurl{http://dx.doi.org/1234.12/Doe.DOI.34}

Parallel particle advection refers to a class of particularly challenging data analysis and visualization algorithms. This is due to load balancing sensitivities, strong data dependencies, and computational requirements. For this reason, it also represents a particularly challenging prospect for in situ visualization. In this talk I will first outline efforts in understanding the effect of the tuneable parameters of particle advection algorithms on performance. I will then follow up with recent work in the application of deep learning to the identifying optimal I/O configurations for simulations running at scale. To conclude I will show how the latter may be directly applied to further understand parallel particle advection algorithms.

%\begin{thebibliography}{0}
%\bibitem{dagrep-manual} Schloss Dagstuhl -- Editorial Office,\textsl{The dagrep class}. Schloss Dagstuhl, Germany, 2011.
%\bibitem{dagrep-sample} John Q. Open; Joan R. Access, \textsl{Seminar Sample}, Dagstuhl Reports, 1:1, 1--8, 2011.
%\end{thebibliography}