\abstracttitle{Supporting the Virtual Red Sea Project}
\abstractauthor[Madhusudhanan Srinivasan]{Madhusudhanan Srinivasan (KAUST, \textcolor{red}{email missing})}
\license

%\jointwork{}
%\abstractref[http://dx.doi.org/1234.12/Doe.DOI.34]{Doe, John; Doe, Jane, Sample talk abstracts used for Dagstuhl Reports, Journal of Seminar Documentation, 1:8, pp.~34--78.}
%\abstractrefurl{http://dx.doi.org/1234.12/Doe.DOI.34}

The Virtual Red Sea project at King Abdullah University of Science and Technology (KAUST) aims to build an integrated data-driven modeling system to study and predict the circulation and the climate of the Red Sea. The scientific goal is to understand atmospheric and oceanic circulations and dynamics, 
atmosphere-ocean-biology interactions, transport and dispersion phenomena, and better reconstruction and forecasting. The Visualization Core Lab at KAUST supports this effort by researching and building in situ tools for analysis, verification, diagnosis, risk assessment and decision support. In this talk, we present our tools and workflows to support visualization and analysis of high-fidelity coupled atmospheric, oceanographic and ecological simulations over the Red Sea in Saudi Arabia.

%\begin{thebibliography}{0}
%\bibitem{dagrep-manual} Schloss Dagstuhl -- Editorial Office,\textsl{The dagrep class}. Schloss Dagstuhl, Germany, 2011.
%\bibitem{dagrep-sample} John Q. Open; Joan R. Access, \textsl{Seminar Sample}, Dagstuhl Reports, 1:1, 1--8, 2011.
%\end{thebibliography}