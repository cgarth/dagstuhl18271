\abstracttitle{ECP ALPINE Algorithm Overview}
\abstractauthor[Gunther H. Weber]{Gunther H. Weber (Lawrence Berkeley National Laboratory, ghweber@lbl.gov)}
\license

%\jointwork{}
%\abstractref[http://dx.doi.org/1234.12/Doe.DOI.34]{Doe, John; Doe, Jane, Sample talk abstracts used for Dagstuhl Reports, Journal of Seminar Documentation, 1:8, pp.~34--78.}
%\abstractrefurl{http://dx.doi.org/1234.12/Doe.DOI.34}

The Exascale Computing Project (ECP) ALPINE project develops algorithms for visualization and analysis that will be critical for ECP applications as the dominant analysis paradigm shifts from post hoc (post processing) to in situ (processing data in a code as it is generated). In this talk I provide an overview over the algorithms currently developed in the project: (1) topological analysis; (2) feature-centric analysis; (3) adaptive sampling; and (4) Lagrangian analysis.

%\begin{thebibliography}{0}
%\bibitem{dagrep-manual} Schloss Dagstuhl -- Editorial Office,\textsl{The dagrep class}. Schloss Dagstuhl, Germany, 2011.
%\bibitem{dagrep-sample} John Q. Open; Joan R. Access, \textsl{Seminar Sample}, Dagstuhl Reports, 1:1, 1--8, 2011.
%\end{thebibliography}