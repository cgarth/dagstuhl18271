
\abstracttitle{Software Complexity, Heterogeneity, and User-facing Issues}
\abstractauthor[John Patchett and E. Wes Bethel]{%
John Patchett (Los Alamos National Laboratory, patchett@lanl.gov)\\
E. Wes Bethel (Lawrence Berkeley National Laboratory, ewbethel@lbl.gov)
}
\license

\wgpar{Contributors:} E. W. Bethel, J. Patchett, P. Messmer, M. Parashar, A. Bauer, M. Srinivasan, K. Ono, N. Röber, N. Gauger, J. Bennett, F. Sadlo, M. Dorier, M. Larsen, A. Ribes Cortes

\begin{refsection}

\wgpar{Background and Motivation.}
The group identified three core problems crosscutting software complexity, heterogeneity and user-facing issues: in situ software is complex and hard to use; users are reluctant to adopt new technologies and in particular in situ; and increasing heterogeneity in computational resources, software tools, and use scenarios exacerbates the problem. 

\wgpar{Challenges.}
In situ software is complex and hard to use. 
Coupling in situ software entails coupling two or more codes together, both of which are likely complex software applications in their own rights. 
This coupling is more complex than simply making a library call, because of the consumer-side factors: the in situ method must be correctly configured to produce the desired results and it may itself be a complex, parallel application. 
In all but the simplest of cases, the data models of the simulation code and in situ method will not be identical.
This necessitates a potentially complex data model transformation.

Additional challenges arise from the complexity of the underlying computational platforms.
These challenges are present for all computational endeavors, not just in situ methods. 
There is an increasing complexity and depth to memory hierarchies, and increasing number of cores per processor. 
With this added complexity arise challenges in determining the optimal configuration and placement of processing components in an in situ pipeline: some methods and configurations are best run truly in situ, where data does not move, while in other configurations, moving data to a different node may produce a lower time to solution. 
This set of challenges is compounded when considering portability: a given configuration that may perform well on one HPC platform is unlikely to be the best for a different platform.

Another challenge area is the fact that many users may be unwilling, or reluctant, to adopt and use in situ methods/infrastructure. 
They are concerned about uncertainties around software lifespan and support, the software complexity, coupled with uncertainty about whether the software will solve their problem, whether it computes correct results, and its reliability and potential adverse impact on their high-concurrency, long-running simulation. 
Users are unwilling to invest time and energy into adopting a new technology unless there are clear benefits to them, and they have a well-founded belief that the new technology they adopt will be around in the future for them.

\wgpar{Successes.}
A Eurographics 2016 State-of-the-Art report STAR report~\cite{Bauer16} provides a comprehensive discourse of in situ R\&D that goes back over 20 years. 
Notably, there are a number of established APIs that target the coupling of simulation and in situ visualization codes~\cite{Fabian11, Whitlock11, Ayachit16, Ayachit16a,Docan2012,Larsen17}.
One of the long-standing weaknesses of in situ approaches is the idea they produce results, i.e., images, that reflect a given combination of visualization and rendering parameters, and the collection of output is then not amenable to exploratory analysis.
CINEMA\cite{Ahrens14} is an approach where in situ visualization tools create a ``CINEMA image database'', such that the images may span some range of visualization and/or rendering parameters, and then a post-hoc viewer permits a user to interact with and browse the collection of pre-computed images. 
Other work in this space includes generating topological summaries for higher-level, feature-based analysis and exploration~\cite{Biedert15}. 
%The DataSpaces~\cite{Docan2012} provides more general programming  abstractions and runtime services to support code coupling and in-situ workflows.

Reasons why users might be more motivated to engage with the in situ community is if such engagement provides more value to their project. 
One example is that it may save them user time in examining results and performing planning for future runs~\cite{Patchett18}. 
Another may be that it enables them to see and study new science hidden in their data, features that had previously gone undiscovered due to limitations of the I/O bottleneck~\cite{Rubel16}, or to perform complex simulation/analysis couplings where the analysis methods consist of combinations of lower-resolution surrogate model codes and advanced geometric/topological analysis methods~\cite{Morozov16}. 
Helping users to become better aware of the technologies and how they might be used is an important part of the landscape, as well. 

\medskip\noindent
\textbf{\sffamily Research Questions.} 
From a resource standpoint, an open and ongoing challenge is coping with increasing heterogeneity of both software infrastructure and the underlying computational platforms. 
On the one hand, users can benefit from a growing collection of tools that provide new, advanced analytics methods. 
On the other hand, finding ways to effectively leverage them within familiar toolchains (integration) and on evolving computational platforms is a challenge. 
Users are also concerned with the reliability of these methods, in terms of correctness of results, along with the robustness of their execution.

A longer-term research question concerns how to minimize the intrusion into the simulation code of the interface code necessary to couple the data producer and consumer. 
While there are efforts that aim at providing portability across numerous in situ backends, the fact that even minimal instrumentation code is required in the simulation impedes adoption. 
It has been suggested that finding ways to eliminate the need for any explicit instrumentation code would represent a significant step forward in this regard. 
Yet, whether this is feasible in production codes remains an open question, chiefly because both data producers (simulations) and consumers (in situ analysis/visualization) are highly complex, massively parallel applications. 

A difficult problem is determining the optimal ratio of simulation to analysis resources, since they are performing different types of computations, along with taking into account the cost of data movement between ranks. 
Helping users to determine optimal configurations of concurrency and placement of these computational components is an open research question, and entails interactions with researchers in other, related fields, such as operating systems/runtime.

Scientists are faced with an increasingly complex problem: computational platforms grow more powerful, but the I/O fabric grows in capacity much more slowly, which results in a debilitating imbalance between their ability to compute data and their ability to store it for analysis. 
One result of this imbalance is the loss of science, where important features in data are not discovered. 
The primary benefit of in situ methods in computational science is the ability to perform better science, and to do so more productively. 
Better science results from having access to full spatio-temporal resolution data for the purpose of knowledge discovery. 
Productivity increases when there is faster turnaround between hypothesis formation, experiment, and analysis of results. 
Reducing complexity, including lowering barriers to use, will result in increased user demand and adoption. 

%\begin{thebibliography}{0}
%    \bibitem{Ahrens14}
%    J. Ahrens et al.:
%    \textsl{An image-based approach to extreme scale in situ visualization and analysis} 
%    In Proceedings of the International Conference for High Performance Computing, Networking, Storage and Analysis. IEEE Press, 2014.
%    %
%    \bibitem{Ayachit16a}
%    U. Ayachit, B. Whitlock, M. Wolf, B. Loring, B. Geveci, D. Lonie, and E. Wes Bethel:
%    \textsl{The SENSEI Generic In Situ Interface}, 
%    In Proceedings of In Situ Infrastructures for Enabling Extreme-scale Analysis and Visualization (ISAV 2016, http://doi.org/10.1109/ISAV.2016.13, 2016.
%    %
%    \bibitem{Ayachit16b} 
%    U. Ayachit, A. Bauer, E. P. N. Duque, G. Eisenhauer, N. Ferrier, J. Gu, K. Jansen, B. Loring, Z. Lukić, S. Menon, D. Morozov, P. O’Leary, M. Rasquin, C. P. Stone, V. Vishwanath, G. H. Weber, B. Whitlock, M. Wolf, K. J. Wu, and E. W. Bethel:
%    \textsl{Performance Analysis, Design Considerations, and Applications of Extreme-scale In Situ Infrastructures}
%    In Proceedings of the International Conference for High Performance Computing, Networking, Storage and Analysis, 2016.
%    %
%    \bibitem{Bauer16}
%    A. C. Bauer, H. Abbasi, J. Ahrens, H.  Childs, B. Geveci, S. Klasky, K. Moreland et al.: 
%    \textsl{In situ methods, infrastructures, and applications on high performance computing platforms}
%    In Computer Graphics Forum, 35(3):577-597. 2016.
%    %
%    \bibitem{Biedert15}
%    T. Biedert and C. Garth:
%    \textsl{Contour tree depth images for large data visualization}
%    In Proceedings of the EG Symposium on Parallel Graphics and Visualization, 2015.
%    %
%    \bibitem{Fabian2011}% 
%    N. Fabian, K. Moreland, D. Thompson, A. C. Bauer, P. Marion, B. Geveci, M. Rasquin, K. E. Jansen: 
%    \textsl{The Paraview Coprocessing Library: A Scalable, General Purpose In Situ Visualization Library},
%    In Proceedings of the IEEE Symposium on Large-Scale Data Analysis and Visualization (LDAV), pp. 89–96, 2011.
%    %
%    \bibitem{Patchett2018}
%    J. Patchett and J. Ahrens:
%    \textsl{Optimizing Scientist Time through In Situ Visualization and Analysis},
%    IEEE Computer Graphics and Applications 38(1):119-127, 2018.
%    %
%    \bibitem{Larsen17}
%    M. Larsen, J. Ahrens, U. Ayachit, E. Brugger, H. Childs, B. Geveci, and C. Harrison:
%    \textsl{The ALPINE In Situ Infrastructure: Ascending from the Ashes of Strawman}
%    In Proceedings of In Situ Infrastructures for Enabling Extreme-Scale Analysis and Visualization (ISAV 2017), 2017. 
%    %
%    \bibitem{Morozov16}
%    D. Morozov and Z. Lukic: 
%    \textsl{Master of Puppets: Cooperative Multitasking for In Situ Processing} In Proceedings of the Symposium on High-Performance Parallel and Distributed Computing (HPDC), 2016.
%    %
%    \bibitem{Ruebel2016}
%    O. Rübel, B. Loring, J. L. Vay, D. P. Grote, R. Lehe, S. Bulanov, H. Vincenti, and E. W. Bethel:
%    \textsl{WarpIV: In Situ Visualization and Analysis of Ion Accelerator Simulations}
%    IEEE Computer Graphics and Applications 36(3):22-–35, 2016.
%    %
%    \bibitem{Whitlock2011}% 
%    B. Whitlock, J. M. Favre, J. S. Meredith:
%    \textsl{Parallel In Situ Coupling of Simulation with a Fully Featured Visualization System}
%    In Proceedings of the EG Symposium on Parallel Graphics and Visualization, pp. 101–109, 2011.
%\end{thebibliography}

\printbibliography	
\end{refsection}


