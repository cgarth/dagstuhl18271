\abstracttitle{Exascale Enables / Requires Different Data}
\abstractauthor[Bernd Hentschel and Dave Pugmire]{%
Bernd Hentschel (RWTH Aachen University)\\
Dave Pugmire (Oak Ridge National Laboratory)
}
\license

\wgpar{Contributors:} A. Bauer, E. W. Bethel, P.-T. Bremer, M. Dorier, M. Hadwiger, C. Hansen, K. Heitmann, I. Hotz, J. Krueger, J. Patchett, K. Moreland, K. Ono, T. Peterka, D. Pleiter, B. Raffin, A. Ribes Cortes, N. Rober, U. R\"ude, F. Sadlo, H-W Shen, M. Srinivasan, G. Weber, R. Westermann, H. Yu

\begin{refsection}

\wgpar{Background and Motivation.}
Approaching the exascale will profoundly change the way we handle data within integrated, in situ systems. 
Specific problems arise with respect to data models for both input and output data, such as increased resolution and/or the creation of large-scale ensembles, the inclusion of (streaming) experimental data, and the reliable execution of computational experiments.

First, driven by energy constraints, data movement will become a key bottleneck. 
Addressing this problem will require the development of shared data models that can be used for both simulation and visualization/analysis alike. 
This is problematic in the sense that today’s formats are typically optimized for either the former or the latter. 

Second, scientists will use next generation machines in different ways. 
Increasing spatio-temporal resolution will make future simulation results inherently multi-scale. 
Yet, the resolution of output displays is bounded by human perceptual limits.
Beyond increased resolution, it is possible to use the system to generate large-scale ensembles by running hundreds or thousands of smaller-scale models in parallel. 
The effective summarization of ensembles including uncertainty quantification is yet unsolved. 

Third, handling data from large-scale experiments will spawn additional requirements. 
This data might be streamed in real-time, e.g., from high-energy physics experiments thus requiring real-time processing. 

Finally, we anticipate that effectively executing computational experiments in the exascale regime will require a constant monitoring of system status and performance data in order to cope with aspects of fault tolerance and optimal resource usage.

While all these problems do have an in situ perspective, we are aware that several aspects are not genuinely driven by the in situ paradigm; in fact, topics like ensemble visualization and uncertainty quantification have been worked on outside the context of large-scale data visualization for years.

\wgpar{Successes.}
To date, several efforts target the exchange of data between simulation codes and visualization exchange.
From a conceptual point of view, theses efforts have in common that they try to provide a generic data model or an interface for the transformation from one model into another one.
Conduit~\cite{Larsen17}, ADIOS~\cite{Lofstead08}, and SENSEI~\cite{Ayachit16} broadly fall into the first category, while the latter comprises, e.g., LibSim~\cite{Whitlock11} and Catalyst~\cite{Fabian11}.

A specific approach to address the need for common data structures are block-structured Adaptive Mesh Refinement (AMR) data~\cite{Berger89}.
AMR uses a hierarchy of axis-aligned rectilinear grids, which are interchangeably called either boxes, patches, or subgrids. 
These form the building blocks that represent the domain. These grids are ordered in a hierarchy of levels having increasing resolution, where data in finer levels replaces those in coarser levels. 
In turn, this can enable multi-resolution simulations such  as mixed atomistic-continuum simulation of materials and multiscale cosmology
simulations.
Initial work on AMR visualization focused on converting AMR data to suitable conventional representations and then visualizing these, e.g.,~\cite{Norman99}.
While these first attempts converted the data into an unstructured mesh, later work focused on maintaining the AMR structure for direct and indirect volume rendering \cite{Wang19,Wald17,Weber12} with a focus on defining smooth interpolation schemes, extracting isosurfaces without discontinuities at hierarchy level boundaries. and efficient parallel implementations.

With respect to workflow systems, current implementations like Pegasus and Flux take a higher-level view of the data and provide a graph-based representation of the operations to be performed. These systems manage the dependencies, execution and resilience of executing the sequence of specified operations.

\wgpar{Research Questions.} 
The aforementioned challenges directly translate into the following set of research questions.

\begin{enumerate}
    \item With regard to data models, we need to analyze and understand what data models are available and which access patterns they facilitate. 
    A profound understanding of predominant access patterns will help us develop new data models and implementations. 
    Eventually, these should substitute existing models wherever feasible in order to increase data re-use and minimize energy-intensive data copies and/or data movement.
    \item Effectively reducing data will require a way to transfer the successes of feature-based visualization from the post hoc world into an in situ setting. When dealing with multi-scale data, we need to research representations that allow for a fluid exploration of data at different scales. 
    \item Analogously, there are many open questions regarding good abstractions and/or summarizations of ensemble data. 
    One can ask in how far good ensemble abstractions help optimize the setup of follow-up computational experiments, e.g., by providing information on parts of the parameters space that need to be sampled more densely?
    \item The execution of large-scale computational experiments begs the question how much of the process can actually be described in a (semi-)formal fashion. 
    Such a description could be used in order to reduce the burden on users; they would only describe what they need, leaving the questions of how, where, and when to a runtime system. 
    The question then is how such descriptions should look like?
\end{enumerate}

\noindent
We note that most of these questions are not specific to the exascale regime. 
Yet, while pre-exascale setups will still allow some leeway, e.g., in terms of the level of integration, the arrival of exascale machines will put in situ techniques to their ultimate test.


%\begin{thebibliography}{0}
%\bibitem{Conduit} 
%... :
%\textsl{ToDo -- Conduit}, 
%.
%%
%\bibitem{ADIOS}% 
%J. F. Lofstead, S. Klasky, K. Schwan, N. Podhorszki, C. Jin: 
%\textsl{Flexible IO and Integration for Scientific Codes Through the Adaptable IO System (ADIOS)},
%In Proceedings of the ACM International Workshop on Challenges of Large Applications in Distributed Environments, pp. 15–24, 2008.
%%
%\bibitem{SENSEI}%
%U. Ayachit, B. Whitlock, M. Wolf, B. Loring, B. Geveci, D. Lonie, and E. Wes Bethel:
%\textsl{The SENSEI Generic In Situ Interface}, 
%In Proceedings of In Situ Infrastructures for Enabling Extreme-scale Analysis and Visualization (ISAV 2016, http://doi.org/10.1109/ISAV.2016.13, 2016.
%%
%\bibitem{LibSim}% 
%B. Whitlock, J. M. Favre, J. S. Meredith:
%\textsl{Parallel In Situ Coupling of Simulation with a Fully Featured Visualization System}
%In Proceedings of the EG Symposium on Parallel Graphics and Visualization, pp. 101–109, 2011.
%%
%\bibitem{Catalyst}% 
%N. Fabian, K. Moreland, D. Thompson, A. C. Bauer, P. Marion, B. Geveci, M. Rasquin, K. E. Jansen: 
%\textsl{The Paraview Coprocessing Library: A Scalable, General Purpose In Situ Visualization Library},
%In Proceedings of the IEEE Symposium on Large-Scale Data Analysis and Visualization (LDAV), pp. 89–96, 2011.
%%
%\bibitem{Berger1989}%
%M. Berger and P. Colella. 
%\textsl{Local Adaptive Mesh Refinement for Shock Hydrodynamics.} 
%Journal of Computational Physics, 82:64–84, 1989.
%%
%\bibitem{Norman1999}% 
%M. L. Norman, J. M. Shalf, S. Levy, and G. Daues: 
%\textsl{Diving Deep: Data Management and Visualization Strategies for Adaptive Mesh Refinement Simulations.} 
%Computing in Science and Engineering, 1(4):36–47, 1999.
%%
%\bibitem{Wald2017}% 
%I. Wald, C. Brownlee, W. Usher, A. Knoll: 
%\textsl{CPU Volume Rendering of Adaptive Mesh Refinement Data},
%In Proceedings of the ACM SIGGRAPH Asia Symposium on Visualization 2017.
%%
%\bibitem{Wang2018}% 
%F. Wang, I. Wald, Q. Wu, W. Usher, C. R. Johnson: 
%\textsl{CPU Iso-surface Ray Tracing of Adaptive Mesh Refinement Data}, 
%to appear - IEEE VIS 2018
%%
%\bibitem{Weber2012}%
%G. H. Weber, H. Childs, and J. S. Meredith: 
%\textsl{Efficient Parallel Extraction of Crack-free Isosurfaces from Adaptive Mesh Refinement (AMR) Data},
%In Proceedings of IEEE Symposium on Large Data Analysis and Visualization (LDAV) pp. 31–38, 2012.
%\end{thebibliography}

\printbibliography
\end{refsection}
